\documentclass[12pt,twoside,a4paper]{article}

%nach folgender Quelle wurden die kommenden Dokument-Einstellungen gesetzt:
%http://latex.hpfsc.de/content/latex_tutorial/umlaut_deutsch/

% deutsche Silbentrennung
\usepackage[ngerman]{babel}
% bez�glich deutschen Umlauten
\usepackage[ansinew]{inputenc}


\usepackage{graphicx}
\usepackage{color}
\definecolor{mygreen}{rgb}{0,0.6,0}

\usepackage{listings}
\lstset{
language=C,
commentstyle=\color{mygreen},
numbers=left,
tabsize=2,	
frame=single
}


\usepackage{subfigure}
\usepackage{hyperref}

\begin{document}


%Titelseite (nach folgender Quelle im Wikipedia nachgebaut: https://de.wikibooks.org/wiki/LaTeX/_Eine_Titelseite_erstellen): 
\begin{titlepage}
	\centering
	%\includegraphics[width=0.15\textwidth]{example-image-1x1}\par\vspace{1cm}
	{\scshape\LARGE hf-ict \par}
	\vspace{1cm}
	{\scshape\Large Diplomarbeit\par}
	\vspace{1.5cm}
	{\huge\bfseries Kassenzettelverwaltung mit iOS-App\par}
	\vspace{2cm}
	{\Large\itshape Timo D�rflinger\par}
	

% Bottom of the page
	{\large \today\par}
\end{titlepage}



% Inhaltsverzeichnis anzeigen
\tableofcontents
% Kapitel soll auf n�chster Seite beginnen
\newpage


\section{Einleitung} 

In meiner Diplomarbeit erstelle ich eine App zur Kassenzettel-Verwaltung. F�r meinen Auftraggeber und mich pers�nlich sind die Kassenzettel verlorene Daten, ausser man verwendet Sie zur pers�nlichen Analyse bzw. zum F�hren eines Haushaltsbuchs. Daher erstelle ich in meiner Diplomarbeit eine iOS-App, mit welcher ein Kassenzettel fotografiert werden kann. Aus diesem erstellten Abbild werden dann gewisse Werte bzw. Daten mit Hilfe eines bereits bestehenden OCR-Frameworks ausgelesen und zur weiteren Verarbeitung zwischengespeichert. Hier kann also unter anderem der Gesamtbetrag ausgelesen und mit einem gesetzten monatlichen Budget verrechnet werden. Dies soll dann auch kategorisiert werden k�nnen. Daraus ist ersichtlich, was in dem laufenden Monat bereits in den verschiedenen Kategorien ausgegeben wurde. Die Kategorien k�nnen jeweils individuell erstellt werden. 


\newpage














\section{Quellenverzeichnis}

Erkl�rung:
\url{https://www.... Link ...}







\section{Bilderverzeichnis}



\end{document}









